\documentclass[conference]{IEEEtran}
\IEEEoverridecommandlockouts
% The preceding line is only needed to identify funding in the first footnote. If that is unneeded, please comment it out.
\usepackage{cite}
\usepackage{amsmath,amssymb,amsfonts}
\usepackage{algorithmic}
\usepackage{graphicx}
\usepackage{textcomp}
\usepackage{xcolor}
\def\BibTeX{{\rm B\kern-.05em{\sc i\kern-.025em b}\kern-.08em
    T\kern-.1667em\lower.7ex\hbox{E}\kern-.125emX}}
\begin{document}

\title{
title
}
\author{\IEEEauthorblockN{Wenjie He}
\IEEEauthorblockA{\textit{Institute of Neuroinformatics} \\
\textit{University of Zurich and ETHz}\\
Zurich, Switzerland\\
he_wenjie@outlook.com}
\and
\IEEEauthorblockN{Given Name Surname}
\IEEEauthorblockA{\textit{dept. name of organization (of Aff.)} \\
\textit{name of organization (of Aff.)}\\
City, Country \\
email address or ORCID}
}

\maketitle

\begin{abstract}
% background - where MS appears, in tilting train / airplane
Motion sickness can be found in various scenarios,

% Current MS experiments - in canbin simulator / MSI curve, frequency weighting
It is found that motion sickness can be induced by low frequency movements and with simulator experiments, a frequency weighting curve for vertical accelerations on motion sickness severity is established in ISO. Further simulator experiments are also conducted for horizontal acceleration, and combination of horizontal and roll movement.

% hypothesis and models - conflict / SV, SVH, CE
It is supposed that motion sickness is induced by the conflict of abundant information (including visual, vestibular signals) perceived by the central nervous system.
%
It is proposed that the conflict between actual and subjective vertical direction can explain the severity of motion sickness and a subjective vertical conflict model (SV model) is developed. Furthermore, subjective vertical and horizontal conflict model (SVH model) is developed based on SV model. 
%
It is also found that combination of head tilt and yaw rotation movement induced Coriolis Effect can also lead to disorienting and nauseating feelings.

% a new combination of different model - a new approch to consider the influence of combination of linear and rotational movement
However, current established model can only be used to investigate pure linear or pure rotational movements.
In this work, a model combining the SVH model and Coriolis Effect (CE\_SVH model) is built trying to help quantify the effect of combining linear and rotational movements on motion sickness.

% active tilting strategy development - time delay and compensation - based on model
Based on the result of the model, vehicle active tilting behaviour concluding lateral acceleration compensation and compensation delay is analysed and compared with experimental results. 

\end{abstract}

\begin{IEEEkeywords}
	motion sickness, Coriolis effect, subjective vertical conflict, subjective horizontal conflict, active tilting
\end{IEEEkeywords}

\section{Introduction}

\section{Existing models}

\section{Model combination}

\section{Active tilting strategy analysis}

\section*{Acknowledgment}

The preferred spelling of the word ``acknowledgment'' in America is without 
an ``e'' after the ``g''. Avoid the stilted expression ``one of us (R. B. 
G.) thanks $\ldots$''. Instead, try ``R. B. G. thanks$\ldots$''. Put sponsor 
acknowledgments in the unnumbered footnote on the first page.

\section*{References}

\begin{thebibliography}{00}
\bibitem{b1} G. Eason, B. Noble, and I. N. Sneddon, ``On certain integrals of Lipschitz-Hankel type involving products of Bessel functions,'' Phil. Trans. Roy. Soc. London, vol. A247, pp. 529--551, April 1955.
\bibitem{b2} J. Clerk Maxwell, A Treatise on Electricity and Magnetism, 3rd ed., vol. 2. Oxford: Clarendon, 1892, pp.68--73.
\bibitem{b3} I. S. Jacobs and C. P. Bean, ``Fine particles, thin films and exchange anisotropy,'' in Magnetism, vol. III, G. T. Rado and H. Suhl, Eds. New York: Academic, 1963, pp. 271--350.
\bibitem{b4} K. Elissa, ``Title of paper if known,'' unpublished.
\bibitem{b5} R. Nicole, ``Title of paper with only first word capitalized,'' J. Name Stand. Abbrev., in press.
\bibitem{b6} Y. Yorozu, M. Hirano, K. Oka, and Y. Tagawa, ``Electron spectroscopy studies on magneto-optical media and plastic substrate interface,'' IEEE Transl. J. Magn. Japan, vol. 2, pp. 740--741, August 1987 [Digests 9th Annual Conf. Magnetics Japan, p. 301, 1982].
\bibitem{b7} M. Young, The Technical Writer's Handbook. Mill Valley, CA: University Science, 1989.
\end{thebibliography}

\end{document}
